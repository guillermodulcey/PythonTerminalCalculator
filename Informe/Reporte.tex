\documentclass[10pt]{report}
\usepackage{graphicx}
\usepackage{verbatim}
\addtolength{\textwidth}{4cm}
\addtolength{\hoffset}{-2cm}
\topmargin -0.30cm

\begin{document}

\begin{titlepage}
	\centering
	\includegraphics[width=0.25\textwidth]{LogoCinvesHeader.png}\par\vspace{2cm}
	{\scshape\LARGE Centro de Investigación y de Estudios Avanzados del IPN\par}
	\vspace{2cm}	
	{\huge\bfseries Proyecto 1. Evaluador de expresiones polinomiales \par}
	\vspace{1.5cm}
	{\scshape\Large Tecnologías de la programación\par}
	\vspace{2cm}
	{\Large Miembros del equipo: \\Hernán Guillermo Dulcey Morán \\Karla Jacquelin Guzmán Sánchez\par}
	\vfill

	{\large 29 de Octubre de 2019 \par}
\end{titlepage}

\section*{1. Introducción}
Se requiere implementar una calculadora que sea capaz de evaluar expresiones polinomiales. El proyecto basa su funcionamiento en expresiones regulares implementadas en el lenguaje de programación Python.

\section*{2. Requerimientos}
El evaluador de expresiones polinomiales debe ser capaz de evaluar, reconocer y ejecutar expresiones matemáticas que incluyan o no alguna función implementada en el evaluador. A continuación se enlistan las funciones que estarán disponibles para el usuario.

\begin{itemize}
\item Suma
\item Resta
\item Multiplicación
\item División
\item Funciones trigonométricas
\begin{comment}
\begin{itemize}
\item Seno
\item Coseno
\item Tangente
\item Secante
\item Cosecante
\item Cotangente
\end{itemize}
\end{comment}
\item Funciones hiperbólicas
\begin{comment}
\begin{itemize}
\item Seno hiperbólico
\item Coseno hiperbólico
\item Tangente hiperbólico
\item Secante hiperbólico
\item Cosecante hiperbólico
\item Cotangente hiperbólico
\end{itemize}
\end{comment}
\item Funciones trigonométricas inversas
\begin{comment}
\begin{itemize}
\item $Seno^{-1}$
\item $Coseno^{-1}$
\item $Tangente^{-1}$
\end{itemize}
\end{comment}
\item Logaritmo natural
\item Logaritmo base 2 
\item Logaritmo base 10
\item Raíz cuadrada
\item Módulo
\end{itemize}

De la misma forma, el evaluador deberá aceptar expresiones de polinomios de la forma:

\begin{center}
$a_m x^n + a_{m-1} x^{n-1}+ a_{m-2} x^{n-2}+ a_{m-3} x^{n-3} + ... + a_1 x + a_0$
\end{center}

Donde $a_i$, $x$ y $m$ son valores en ${\rm I\!R}$ que corresponen a los coeficientes, variables independientes y exponentes respectivamente. La expresión polinomial también deberá aceptar más de una variable independiente.

La calculadora será capaz de procesar cualquier operación mencionada en los requerimientos siempre y cuando lleve la sintaxis adecuada.

\section*{3. Manual}
La calculadora es capaz de resolver expresiones sin la necesidad de parentesis pero se recomienda hacer uso de estos para tener un mejor orden de la expresión.

La calculadora puede asignar valores a variables de la forma: x=1. En este caso se le ha asignado el valor de 1 a la variable x, por lo tanto en las siguientes expresiones que contengan la variable x, se reconocerá este valor y será reemplazado para resolver la ecuación. En caso de que el valor no haya sido asignado, la ecuación no puede resolverse.

El nombre de la variable siempre debe iniciar por una letra, pero apartir de ahí puede tomar cualquier valor excepto valores que hagan que el nombre de la variable coincida con una palabra reservada. La calculadora tiene las siguientes palabras reservadas, las cuales no pueden ser usadas para asignación de variables:
\begin{enumerate}
    \item ln : logaritmo natural
    \item log : logaritmo con base
    \item sqrt : raíz cuadrada
    \item sen,cos,tan,cot,csc,sec : funciones trigonometricas
    \item asen,acos,atan : funciones trigonometricas inversas
    \item senh,cosh,tanh,coth,csch,sech : funciones hiperbólicas
    \item asenh,acosh,atanh : funciones hiperbólicas inversas.
\end{enumerate}
Así como tambien existen caracteres especiales:
\begin{enumerate}
    \item * : operador de multiplicación
    \item / : operador de división
    \item \% : operador de módulo
    \item + : operador de suma
    \item - : operador de resta
    \item \textbf{\^} : operador de potencia
    \item \textbf{.} : separador de decimales
    \item \textbf{,} : separador de base de logaritmo con base
    \item = : operador de asignación
\end{enumerate}
Cabe resaltar que la calculadora maneja el orden PEMDAS para las operaciones (Parentesis, exponente, multiplicación, división, adición y substracción). Las operaciones de logaritmo y trigonometricas son consideradas de orden aún mayor a las mencionadas anteriormente.

A continuación se describe la sintaxis que debe seguir el usuario para utilizar las funciones de manera correcta.

\begin{itemize}
\item Número entero: 3 

\item Número decimal: 2\textbf{.}3

\item Suma: 2 \textbf{+} 3

Respuesta esperada: 5

\item Resta: 2 \textbf{-} 3

Respuesta esperada: -1

\item Multiplicación 2 \textbf{*} 3

Respuesta esperada: 6

\item División: 9 \textbf{/} 3

Respuesta esperada: 3

\item Módulo: 10 \textbf{\%} 3

Respuesta esperada: 1

\item Potencia: 2 \textbf{\^} 2

Respuesta esperada: 4

\item Asignación: x \textbf{=} 1

Nota: Se hace distinción entre mayúsculas y minúsculas. \textbf{x} NO es equivalente a \textbf{X}.

\item Agrupación

\textbf{(}$2+2$\textbf{)}$*4$

Respuesta esperada: 16

\item Raíz cuadrada: \textbf{sqrt} 4 ó \textbf{SQRT} 4

Respuesta esperada: 2

\item Logaritmos:

\begin{itemize}
\item Logaritmo natural: \textbf{ln} 1 ó \textbf{LN} 1

Respuesta esperada: 0

\item Logartimo con base: \textbf{log} número\textbf{,} base ó \textbf{LOG} 5\textbf{,}10

Ejemplo: Logaritmo base 10 de 5. \textbf{log} 5\textbf{,}10

Respuesta esperada: 0.6989700043360187
\end{itemize}

\item Funciones trigonométricas

\begin{itemize}
\item Seno: \textbf{sen} 90 ó \textbf{SEN} 90
\item Coseno: \textbf{cos} 90 ó \textbf{COS} 90
\item Tangente: \textbf{tan} 90 ó \textbf{TAN} 90
\item Secante: \textbf{sec} 90 ó \textbf{SEC} 90
\item Cosecante: \textbf{csc} 90 ó \textbf{CSC} 90
\item Cotangente: \textbf{cot} 90 ó \textbf{COT} 90
\end{itemize}

\item Funciones trigonómetricas inversas
\begin{itemize}
\item Arcoseno: \textbf{asen} 90 ó \textbf{ASEN} 90
\item Arcocoseno: \textbf{acos} 90 ó \textbf{ACOS} 90
\item Arco tangente: \textbf{atan} 90 ó \textbf{ATAN} 90
\end{itemize}

\item Funciones hiperbólicas

\begin{itemize}
\item Seno hiperbólico: \textbf{senh} 0.90 ó \textbf{SENH} 0.90
\item Coseno hiperbólico: \textbf{cosh} 0.90 ó \textbf{COSH} 0.90
\item Tangente hiperbólico: \textbf{tanh} 0.90 ó \textbf{TANH} 0.90
\item Secante hiperbólico: \textbf{sech} 0.90 ó \textbf{SECH} 0.90
\item Cosecante hiperbólico: \textbf{csch} 0.90 ó \textbf{CSCH} 0.90
\item Cotangente hiperbólico: \textbf{coth} 0.90 ó \textbf{COTH} 0.90
\end{itemize}

\item Funciones hiperbólicas inversas

\begin{itemize}
\item Arcoseno hiperbólico: \textbf{asenh} 0.90 ó \textbf{ASENH} 0.90
\item Arcocoseno hiperbólico: \textbf{acosh} 0.90 ó \textbf{ACOSH} 0.90
\item Arcotangente hiperbólico: \textbf{atanh} 0.90 ó \textbf{ATANH} 0.90
\end{itemize}

\end{itemize}

Nota: Todas las operaciones o funciones mencionadas pueden hacer uso de paréntesis para agrupar sus componentes si el usuario lo desea. Ejemplo: \textbf{ln} \textbf{(}5,10\textbf{)} es equivalente a \textbf{ln} 5\textbf{,}10 .

\section*{4. Conclusión}
Las expresiones regulares nos ayudan a entender la sintaxis de una expresión y gracias a eso podemos añadir lógica al resultado de dicha expresión. Así como se analizaron sentencias simples, con nuevos procedimientos se podrían realizar el analisis y obtener el resultado de operaciones más complejas. 

\end{document}